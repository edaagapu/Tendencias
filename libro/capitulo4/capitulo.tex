\chapter{Planificaci\'on del Proyecto de Software}
En el presente cap\'itulo se brinda al cliente un modelado inicial del software con todos los requerimientos anteriormente descritos, puliendo de esa manera los criterios mediante los cuales se trabaja, el tiempo estimado del desarrollo del proyecto y los costos preliminares.

\section{Alcances Proyecto de Software}
\begin{itemize}
	\item 
\end{itemize}
\section{Objetivos Proyecto de Software}
\begin{itemize}
	\item 
\end{itemize}
\section{Metodolog\'ia Proyecto de Software}
La metodolog\'ia usada para el proyecto es SCRUM

\section{Determinaci\'on de Recursos del Proyecto de Software}
\subsection{Recursos Humanos}
\begin{table}[h!]
	\begin{center}
		\begin{tabular}{|p{5cm} |p{5cm}|p{4cm}|} 
			\hline \textbf{Nombre} & \textbf{Perfil} & \textbf{Rol} \\
			\hline Edwin Garcia & Ingeniero de Sistemas & Desarrollador \\ 
			\hline Brian Rodr\'iguez & Ingeniero de Sistemas & Desarrollador \\
			\hline
		\end{tabular}
		\caption{Equipo de Desarrollo}
		\label{teamDevelop}
	\end{center}
\end{table}
%(Equipo de trabajo, perfiles y roles)
\subsection{Recursos Tecnologicos}
\subsubsection*{Hardware}
Se usaran dos computadores para el desarrollo del aplicativo. \\
\begin{table}[h!]
	\begin{center}
		\begin{tabular}{|p{7cm} |p{7cm}|} 
			\hline \textbf{Marca} &  HP 14\\
			\hline \textbf{Procesador} &  Intel Core i5\\
			\hline \textbf{RAM} &  8 Gb\\
			\hline \textbf{Disco Duro} &  700 Gb\\
			\hline \textbf{Sistema Operativo} &  GNU/Linux (Linux Mint)\\
			\hline
		\end{tabular}
		\caption{Referencias del Primer Computador}
		\label{equipoU}
	\end{center}
\end{table}

\begin{table}[h!]
	\begin{center}
		\begin{tabular}{|p{7cm} |p{7cm}|} 
			\hline \textbf{Marca} &  <> \\
			\hline \textbf{Procesador} &  <>\\
			\hline \textbf{RAM} &  <>\\
			\hline \textbf{Disco Duro} &  <>\\
			\hline \textbf{Sistema Operativo} &  <>\\
			\hline
		\end{tabular}
		\caption{Referencias del Segundo Computador}
		\label{equipoD}
	\end{center}
\end{table}
\subsubsection*{Software}
Se usar\'a el IDLE de Eclipse para el desarrollo del aplicativo (Mediante el lenguaje JAVA), de igual forma se usará Enterprise Architect 11 para la realizaci\'on del modelamiento en software del aplicativo.
\subsubsection*{Base de Datos}
Se usar\'a como motor de persistencia PgAdmin III, y a su vez el lenguaje mediante el cual se representar\'a la base de datos ser\'a PostgreSQL, haciendo conexi\'on con Eclipse.
\subsubsection*{Comunicaciones}

\subsubsection*{Ingenier\'ia Web}

%Hardware, Software, Aplicaciones y BD, Comunicaciones, Ingeniería Web

\subsection{Recursos Financieros} 
%Pendiente por hacer por Edwin
%(Estimación – Relación B/C)
\section{Cronograma de Actividades}
%Socializarlo para realizarlo
%(Gannt)
