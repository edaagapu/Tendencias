\chapter{Requerimientos}
\section{Identificaci\'on y Justificaci\'on del \'Area funcional}
La organización Drogas RUZVEL presenta falencias a nivel de manejo de inventario y de la misma información de sus medicamentos debido a que la misma organización posee una cantidad muy reducida de empleados los cuales han estado trabajando por un largo tiempo allí y al momento de asignar nuevos empleados, no se tiene un registro del inventario del local ni la información de cada medicamento, lo cual implica un manejo precario de las cantidades existentes y faltantes, así como la necesidad de solicitar ayuda de los mismos empleados antiguos para ofrecer al cliente cualquier información general requerida de un medicamento o similares en cuestiones de la función de los mismos ; sin embargo presentan un manejo de inventario físico a priori, situación que ralentiza los procesos de abastecimiento del local para satisfacer cada una de las necesidades de los posibles clientes, puesto que no se realiza un pedido para adquirir nuevos medicamentos sino hasta que el empleado vea que falten por el pedido de un cliente al cual no pueda atender produciendo pérdidas para la organización. 

Por lo anterior la empresa SofBORe, se encargará de diseñar, desarrollar e implementar un módulo de gestión de inventario en el cual se obtendrá información tanto general como de existencias concerniente a cada medicamento, con el propósito de optimizar tiempo y calidad de servicio del conteo de medicamentos y abastecimiento necesario.

\section{Caracterizaci\'on de los Procesos y Actividades}

Dentro de lo procesos y actividades realizados por la droguería que se llevan a cabo para la la gestión de inventario e información de medicamentos se tienen:
En primera instancia una vista general a cada estante dentro del local para saber más o menos cuantos medicamentos quedan existentes en el lugar, así mismo si se ve ve que hay aproximadamente uno, se procede a escribir en una hoja de papel el nombre del medicamento que se va a agotar para posteriormente realizar un pedido del mismo. 
En segundo lugar el empleado se da cuenta de la escasez del producto cuando un cliente ingresa a la droguería a solicitar un medicamento, hasta que no se es buscado por toda la estantería  y no se encuentra ningún ejemplar del medicamento buscado. 
Por tercer ámbito se tiene que, cuando un cliente llega a pedir información de un medicamento o una recomendación de qué otro podría usar para compensar ese medicamento si no lo puede comprar, sólo se le puede brindar una buena atención e información precisa si se encuentra el empleado experto en la información de cada medicamento que maneja, si es un empleado nuevo, tendrá que solicitar ayuda al personal antiguo si no conoce medicamentos o si se le es complicado recordar los nuevos que han ingresado al local.

Las actividades mencionadas anteriormente son netamente físicas y sin tecnología involucrada allí.


\section{An\'alisis de Requerimientos de Informaci\'on}

Los requerimientos de información para el sistema de gestión de la droguería estarán vinculados a la información encontrada en la base de datos del sistema de registros INVIMA, así como documentación de otras entidades que no se pueda encontrar en los registros mencionados anteriormente.


\section{Obtenci\'on de Requerimientos}
\subsection{Funcionales}

\subsection{No Funcionales}

\subsection{Dominio}

\subsection{Usuarios}

\subsection{Sistema}
