\chapter{Requerimientos}
\section{Identificaci\'on y Justificaci\'on del \'Area funcional}
La organización Drogas RUZVEL presenta falencias a nivel de manejo de inventario y de la misma información de sus medicamentos debido a que la misma organización posee una cantidad muy reducida de empleados los cuales han estado trabajando por un largo tiempo allí y al momento de asignar nuevos empleados, no se tiene un registro del inventario del local ni la información de cada medicamento, lo cual implica un manejo precario de las cantidades existentes y faltantes, así como la necesidad de solicitar ayuda de los mismos empleados antiguos para ofrecer al cliente cualquier información general requerida de un medicamento o similares en cuestiones de la función de los mismos ; sin embargo presentan un manejo de inventario físico a priori, situación que ralentiza los procesos de abastecimiento del local para satisfacer cada una de las necesidades de los posibles clientes, puesto que no se realiza un pedido para adquirir nuevos medicamentos sino hasta que el empleado vea que falten por el pedido de un cliente al cual no pueda atender produciendo pérdidas para la organización. 

Por lo anterior la empresa SofBORe, se encargará de diseñar, desarrollar e implementar un módulo de gestión de inventario en el cual se obtendrá información tanto general como de existencias concerniente a cada medicamento, con el propósito de optimizar tiempo y calidad de servicio del conteo de medicamentos y abastecimiento necesario.

\section{Caracterizaci\'on de los Procesos y Actividades}

Dentro de lo procesos y actividades realizados por la droguería que se llevan a cabo para la la gestión de inventario e información de medicamentos se tienen:
En primera instancia una vista general a cada estante dentro del local para saber más o menos cuantos medicamentos quedan existentes en el lugar, así mismo si se ve ve que hay aproximadamente uno, se procede a escribir en una hoja de papel el nombre del medicamento que se va a agotar para posteriormente realizar un pedido del mismo. 
En segundo lugar el empleado se da cuenta de la escasez del producto cuando un cliente ingresa a la droguería a solicitar un medicamento, hasta que no se es buscado por toda la estantería  y no se encuentra ningún ejemplar del medicamento buscado. 
Por tercer ámbito se tiene que, cuando un cliente llega a pedir información de un medicamento o una recomendación de qué otro podría usar para compensar ese medicamento si no lo puede comprar, sólo se le puede brindar una buena atención e información precisa si se encuentra el empleado experto en la información de cada medicamento que maneja, si es un empleado nuevo, tendrá que solicitar ayuda al personal antiguo si no conoce medicamentos o si se le es complicado recordar los nuevos que han ingresado al local.

Las actividades mencionadas anteriormente son netamente físicas y sin tecnología involucrada allí.


\section{An\'alisis de Requerimientos de Informaci\'on}

Los requerimientos de información para el sistema de gestión de la droguería estarán vinculados a la información encontrada en la base de datos del sistema de registros INVIMA, así como documentación de otras entidades que no se pueda encontrar en los registros mencionados anteriormente.


\section{Obtenci\'on de Requerimientos}
\subsection{Funcionales}

\subsection{No Funcionales}

\subsection{Dominio}

\subsection{Usuarios}
\begin{longtable}[c]{|c|l|l|}
\hline
\textbf{Identificador} & \multicolumn{1}{c|}{\textbf{Descripción}} & \textbf{Prioridad} \\ \hline
\endhead
%
\textbf{RU01} & \begin{tabular}[c]{@{}l@{}}El sistema deberá permitir el registro\\ de los medicamentos.\end{tabular} & \begin{tabular}[c]{@{}l@{}}Alta\end{tabular} \\ \hline
\textbf{RU02} & \begin{tabular}[c]{@{}l@{}}Se podrá generar la tabla de información\\ de forma automática.\end{tabular} & \begin{tabular}[c]{@{}l@{}}Media\end{tabular} \\ \hline
\textbf{RU03} & \begin{tabular}[c]{@{}l@{}}El sistema deberá notificar localmente los \\ medicamentos próximos a vencer.\end{tabular} & \begin{tabular}[c]{@{}l@{}}Alta\end{tabular} \\ \hline
\textbf{RU04} & \begin{tabular}[c]{@{}l@{}}El sistema podrá alertar la necesidad\\ de reabastecer medicamentos.\end{tabular} & \begin{tabular}[c]{@{}l@{}}Media\end{tabular} \\ \hline
\textbf{RU05} & \begin{tabular}[c]{@{}l@{}}El sistema permitirá la actualización de información\\ de cantidades y fecha de vencimiento de cada medicamento.\end{tabular} & \begin{tabular}[c]{@{}l@{}}Alta\end{tabular} \\ \hline
\textbf{RU06} & \begin{tabular}[c]{@{}l@{}}Facilidad de uso: El sistema se diseñará de tal\\ forma que permita al usuario una navegación\\ intuitiva.\end{tabular} & \begin{tabular}[c]{@{}l@{}}Alta\end{tabular} \\ \hline
\textbf{RU07} & \begin{tabular}[c]{@{}l@{}}Seguridad: Las actualizaciones de información\\ solo podrán ser efectuadas por el\\ encargado en turno.\end{tabular} & \begin{tabular}[c]{@{}l@{}}Media-Alta\end{tabular} \\ \hline
\textbf{RU08} & \begin{tabular}[c]{@{}l@{}}Portabilidad: Se podrá acceder al sistema desde\\ cualquier sistema operativo compatible a la base de \\ datos, encontrada en un disco duro externo.\end{tabular} & \begin{tabular}[c]{@{}l@{}}Alta\end{tabular} \\ \hline

\subsection{Sistema}

\begin{longtable}[c]{|l|l|l|l|}
\captionsetup{justification=centering}
\hline
\textbf{Identificador} & \multicolumn{1}{c|}{\textbf{Nombre}} & \textbf{Deriva} & \multicolumn{1}{c|}{\textbf{Descripción}} \\ \hline
\endhead
%
\textbf{RS01} & \begin{tabular}[c]{@{}l@{}}Registro de Medicamentos.\end{tabular} & RU01 & \begin{tabular}[c]{@{}l@{}}Al registrar un medicamento el sistema \\ presentará al usuario un formulario solicitando\\ los siguientes campos:\\ Denominación Comercial\\ Denominación Científica\\ Composición \\  Fecha de Vencimiento\\  Modo de Empleo\\ Tipo Prescripción \\ Empresa \\ Clase\\ Para qué sirve \\ Cantidad Disponible \end{tabular} \\ \hline

\textbf{RS02} & \begin{tabular}[c]{@{}l@{}}Tabla de información\end{tabular} & RU02 & \begin{tabular}[c]{@{}l@{}}Al solicitar la generación de la información de \\ un medicamento, el sistema deberá mostrar\\ la misma tabla encontrada en la base de\\ datos con todos los campos del medicamento\\ solicitado.\end{tabular} \\ \hline

\textbf{RS03} & \begin{tabular}[c]{@{}l@{}}Notificaciones\end{tabular} & RU03 & \begin{tabular}[c]{@{}l@{}}El sistema deberá presentar el nombre \\ comercial y científico del o los medicamentos\\ próximos a vencer, junto a la fecha de\\ dvencimiento y la cantidad de los mismos que\\ tengan esa fecha registrada.\end{tabular} \\ \hline

\textbf{RS04} & \begin{tabular}[c]{@{}l@{}}Abastecimiento\end{tabular} & RU04 & \begin{tabular}[c]{@{}l@{}}Al momento de recibir un cambio de la \\ cantidad de medicamentos menor a 3, el \\ sistema mostrará el nombre comercial y \\ científico del o los medicamentos informando\\ al usuario que debe realizar un pedido de ese \\ medicamento. \end{tabular} \\ \hline

\textbf{RS05} & \begin{tabular}[c]{@{}l@{}}Actualización de Información\end{tabular} & RU05 & \begin{tabular}[c]{@{}l@{}}El sistema permitirá al usuario cambiar los \\ campos de cantidades y fecha de vencimiento \\ de los medicamentos e informará el cambio \\ exitoso mediante una alerta informativa al\\ mismo una vez se haya actualizado la \\ información en la base de datos. \end{tabular} \\ \hline

\textbf{RS06} & \begin{tabular}[c]{@{}l@{}}Interfaz\end{tabular} & RU06 & \begin{tabular}[c]{@{}l@{}}El sistema debe contar con una distribución \\ “inteligente” de las secciones, botones y \\ herramientas de tal forma que se cree una \\ experiencia de navegación en el sistema \\ satisfactoria.\end{tabular} \\ \hline

\textbf{RS07} & \begin{tabular}[c]{@{}l@{}}Seguridad\end{tabular} & RU07 & \begin{tabular}[c]{@{}l@{}}El sistema debe contar con una cuenta de  \\ usuario para el empleado encargado de la \\ gestión de inventario, con el fin de asegurar\\ que los cambios en los campos que se\\ puedan solo puedan realizarse por el\\ encargado en turno. \end{tabular} \\ \hline
