\chapter{Requerimientos}
\section{Identificaci\'on y Justificaci\'on del \'Area funcional}
La organización Drogas RUZVEL presenta falencias a nivel de manejo de inventario y de la misma información de sus medicamentos debido a que la misma organización posee una cantidad muy reducida de empleados los cuales han estado trabajando por un largo tiempo allí y al momento de asignar nuevos empleados, no se tiene un registro del inventario del local ni la información de cada medicamento, lo cual implica un manejo precario de las cantidades existentes y faltantes, así como la necesidad de solicitar ayuda de los mismos empleados antiguos para ofrecer al cliente cualquier información general requerida de un medicamento o similares en cuestiones de la función de los mismos ; sin embargo presentan un manejo de inventario físico a priori, situación que ralentiza los procesos de abastecimiento del local para satisfacer cada una de las necesidades de los posibles clientes, puesto que no se realiza un pedido para adquirir nuevos medicamentos sino hasta que el empleado vea que falten por el pedido de un cliente al cual no pueda atender produciendo pérdidas para la organización. 

Por lo anterior la empresa SofBORe, se encargará de diseñar, desarrollar e implementar un módulo de gestión de inventario en el cual se obtendrá información tanto general como de existencias concerniente a cada medicamento, con el propósito de optimizar tiempo y calidad de servicio del conteo de medicamentos y abastecimiento necesario.

\section{Caracterizaci\'on de los Procesos y Actividades}

Dentro de lo procesos y actividades realizados por la droguería que se llevan a cabo para la la gestión de inventario e información de medicamentos se tienen:
En primera instancia una vista general a cada estante dentro del local para saber más o menos cuantos medicamentos quedan existentes en el lugar, así mismo si se ve ve que hay aproximadamente uno, se procede a escribir en una hoja de papel el nombre del medicamento que se va a agotar para posteriormente realizar un pedido del mismo. 
En segundo lugar el empleado se da cuenta de la escasez del producto cuando un cliente ingresa a la droguería a solicitar un medicamento, hasta que no se es buscado por toda la estantería  y no se encuentra ningún ejemplar del medicamento buscado. 
Por tercer ámbito se tiene que, cuando un cliente llega a pedir información de un medicamento o una recomendación de qué otro podría usar para compensar ese medicamento si no lo puede comprar, sólo se le puede brindar una buena atención e información precisa si se encuentra el empleado experto en la información de cada medicamento que maneja, si es un empleado nuevo, tendrá que solicitar ayuda al personal antiguo si no conoce medicamentos o si se le es complicado recordar los nuevos que han ingresado al local.

Las actividades mencionadas anteriormente son netamente físicas y sin tecnología involucrada allí.


\section{An\'alisis de Requerimientos de Informaci\'on}
Los requerimientos de información para el sistema de gestión de la droguería estarán vinculados a la información encontrada en la base de datos del sistema de registros INVIMA, así como documentación de otras entidades que no se pueda encontrar en los registros mencionados anteriormente.


\section{Obtenci\'on de Requerimientos}
\subsection{Funcionales}
A continuación se pueden observar los requerimientos funcionales: \\
\begin{table}[h!]
	\begin{center}
		\begin{tabular}{| l |p{10cm} | l |p{30cm}|} 
			\hline
			\textbf{Identificador} & \textbf{Descripci\'on} \\
			\hline
		\textbf{RF01} & El empleado debe tener la posibilidad de ver en donde están almacenados los productos.\\ \hline
		\textbf{RF02} & El empleado debe notificar la cantidad que hay de un producto.\\ \hline
		\textbf{RF03} & Se debe tener la capacidad de ver el inventario de la droguería.\\ \hline
		\textbf{RF04} & El administrador podrá generar y eliminar usuarios,\\ \hline
		\textbf{RF05} & El usuario (Ya sea administrador o empleado) debe ser alertado sobre el estado de los medicamentos.\\ \hline
		\end{tabular}
		\caption{Requerimientos Funcionales}
		\label{reqFunc}
	\end{center}
\end{table}

\newpage
\subsection{No Funcionales}
A continuación se pueden observar los requerimientos no funcionales: \\
\begin{table}[h!]
	\begin{center}
		\begin{tabular}{| l |p{10cm} | l |p{30cm}|} 
			\hline
			\textbf{Identificador} & \textbf{Descripci\'on} \\
			\hline
			\textbf{RNF01} & Se debe brindar la seguridad de que no se pueda acceder a la clave de acceso de ningún usuario, desde otro.\\ \hline
			\textbf{RNF02} & Se deben brindar las diferentes capacidades de cada tipo de usuario, sin que estas se entremezclen\\ \hline
			\textbf{RNF03} & El sistema debe brindar las facilidades de uso del programa al usuario.\\ \hline
			\textbf{RNF04} & La curva de aprendizaje del programa se reducirá al mínimo, quitando de esa manera la mayoría de las dificultades a la hora de aprender el manejo del sistema\\ \hline
		\end{tabular}
		\caption{Requerimientos no Funcionales}
		\label{reqNFunc}
	\end{center}
\end{table}
\newpage

\subsection{Dominio}
A continuación se pueden observar los requerimientos del dominio: \\
\begin{table}[h!]
	\begin{center}
		\begin{tabular}{| l |p{10cm} | l |p{30cm}|} 
			\hline
			\textbf{Identificador} & \textbf{Descripci\'on} \\
			\hline
			\textbf{RD01} & El administrador deberá llenar a cabalidad la información de la creación de usuario. \textit{(Identificacion,  nombres, apellidos, rol)}\\ \hline
			\textbf{RD02} & El usuario, deberá notificar al sistema cuando se retire uno o varios productos de la droguería, junto con la fecha en que se retira, para que el inventario se genere de forma correcta.\\ \hline
			\textbf{RD03} & El empleado deberá llenar a cabalidad los datos correspondientes a un nuevo producto. \textit{(Serial, nombre, descripción, fecha de fabricación, fecha de vencimiento, recomendaciones, contraindicaciones, costo)}\\ \hline
		\end{tabular}
		\caption{Requerimientos del Dominio}
		\label{reqDom}
	\end{center}
\end{table}

\newpage
\subsection{Usuario}
A continuación se pueden observar los requerimientos de usuario: \\
\begin{table}[h!]
	\begin{center}
		\begin{tabular}{| l |p{9cm} | l |p{13cm} | c |p{9cm}|} 
			\hline
			\textbf{Identificador} & \textbf{Descripci\'on} & \textbf{Prioridad} \\
			\hline
			\textbf{RU01} & El sistema deberá permitir el registro de los medicamentos. & Alta\\ \hline
			\textbf{RU02} & Se podrá generar la tabla de información de forma automática. & Media\\ \hline
			\textbf{RU03} & El sistema deberá notificar localmente los medicamentos próximos a vencer & Alta\\ \hline
			\textbf{RU04} & El sistema podrá alertar la necesidad de reabastecer medicamentos. & Media\\ \hline
			\textbf{RU05} & El sistema permitirá la actualización de información de cantidades y fecha de vencimiento de cada medicamento. & Alta\\ \hline
			\textbf{RU06} & \textbf{Facilidad de Uso: } El sistema se diseñará de tal forma que permita al usuario una navegación intuitiva. & Alta\\ \hline
			\textbf{RU07} & \textbf{Seguridad: } Las actualizaciones de información solo podrán ser efectuadas por el encargado en turno. & Media-Alta\\ \hline
			\textbf{RU08} & \textbf{Portabilidad:} Se podrá acceder al sistema desde cualquier sistema operativo compatible a la base de datos, encontrada en un disco duro externo. & Media\\ \hline
		\end{tabular}
		\caption{Requerimientos de Usuario}
		\label{reqUsu}
	\end{center}
\end{table}
\subsection{Sistema}
A continuación se pueden observar los requerimientos del sistema: \\
\begin{table}[h!]
	\begin{center}
		\begin{tabular}{| l |p{3cm} | l |p{7cm} | c |p{10cm} | c |p{45cm} | } 
			\hline
			\textbf{Identificador} & \textbf{Nombre} & \textbf{Deriva} & \textbf{Descripci\'on} \\
			\hline
			\textbf{RS01} & Registro de Medicamentos & \textbf{RU01} & Al registrar un medicamento el sistema presentará al usuario un formulario solicitando los siguientes campos:
			\textit{Denominación Comercial, Denominación Científica, Composición, Fecha de Vencimiento, Modo de Empleo, Tipo Prescripción, Empresa, Clase, Para qué sirve, Cantidad Disponible.} 
			 \\ \hline
			 \textbf{RS02} & Tabla de información & \textbf{RU02} & Al solicitar la generación de la información de un medicamento, el sistema deberá mostrar la misma tabla encontrada en la base de datos con todos los campos del medicamento solicitado.\\ \hline
			 \textbf{RS03} & Notificaciones & \textbf{RU03} & El sistema deberá presentar el nombre comercial y científico del o los medicamentos próximos a vencer, junto a la fecha de vencimiento y la cantidad de los mismos que tengan esa fecha registrada.\\ \hline
			 \textbf{RS04} & Abastecimiento & \textbf{RU04} & Al momento de recibir un cambio de la cantidad de medicamentos menor a 3, el sistema mostrará el nombre comercial y científico del o los medicamentos informando al usuario que debe realizar un pedido de ese medicamento.\\ \hline
			 \textbf{RS05} & Actualización de Información & \textbf{RU05} & El sistema permitirá al usuario cambiar los campos de cantidades y fecha de vencimiento de los medicamentos e informará el cambio exitoso mediante una alerta informativa al mismo una vez se haya actualizado la información en la base de datos.\\ \hline
			 \textbf{RS06} & Interfaz & \textbf{RU06} & El sistema debe contar con una distribución “inteligente” de las secciones, botones y herramientas de tal forma que se cree una experiencia de navegación en el sistema satisfactoria.\\ \hline
			 \textbf{RS07} & Seguridad & \textbf{RU07} & El sistema debe contar con una cuenta de usuario para el empleado encargado de la gestión de inventario, con el fin de asegurar que los cambios en los campos que se puedan solo puedan realizarse por el encargado en turno.\\ \hline
		\end{tabular}
		\caption{Requerimientos del Sistema}
		\label{reqSis}
	\end{center}
\end{table}
